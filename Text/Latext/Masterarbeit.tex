\documentclass[a4paper]{article}  

\usepackage[utf8]{inputenc}
\usepackage[ngerman]{babel}	
\usepackage[none]{hyphenat}
\usepackage{listings}
\usepackage{graphicx}
\usepackage{float}
\usepackage[hidelinks]{hyperref}
\usepackage{geometry}

\setlength{\parindent}{0em}
\geometry{a4paper, top=25mm, left=40mm, right=25mm, bottom=30mm,
headsep=10mm, footskip=12mm} 

\title{Masterarbeit}
\author{Jens Grundmann}
\date{\today} 

\begin{document}

	\begin{titlepage}
		\maketitle	
	\end{titlepage}

\tableofcontents
\newpage

	\section{Einleitung}
	Das Ziel dieses Forschungsprojektes ist die Untersuchung der Möglichkeit einer
	Kommunikation von JavaScript mit einer Java API. Die zugrunde liege API ist hier das 
	von Prof. Thomas Schwotzer liegende ScharkNet. \cite{SharkNet01} \\
	
	Diese Brücke soll es ermöglichen genannte API von Mozilla Thunderbird aus anzusprechen.
	Die Gestaltung der Oberfläche in vielen Mozilla Produkten basiert auf einer XML 
	Syntax genannt XUL \cite{Mozilla01}. Dieses wird mit CSS gestaltet und dynamische 
	Inhalt werden über JavaScript programmiert, womit es starke Ähnlichkeiten mit HTML
	hat. Somit würde dies eine Möglichkeit der Gestaltung einer grafischen Oberfläche 
	bieten, die, gegenüber Bibliotheken wie Java Swing, einfacher zu handhaben ist und
	überzeugendere Ergebnisse, im Sinne grafischer Gestaltung, bieten würde. Da es sich
	bei XUL Dateien grundlegend um ein XML Dokument handelt, kann dessen DOM auch mit Bibliotheken 
	wie jQuery \cite{jQuery01} manipuliert werden, was JavaScript die gleichen Möglichkeiten
	wie in einem Webbrowser bietet. Zusätzlich könnten man erwähnen Funktionen der Anwendung,
	wie das senden von Emails in Thunderbird, zu nutzen.

\end{document}
