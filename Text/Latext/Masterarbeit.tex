\documentclass[a4paper]{article}  

\usepackage[utf8]{inputenc}
\usepackage[ngerman]{babel}	
\usepackage[none]{hyphenat}
\usepackage{listings}
\usepackage{graphicx}
\usepackage{float}
\usepackage[hidelinks]{hyperref}
\usepackage{geometry}
\usepackage{wallpaper}

\geometry{
	a4paper, 
	top=25mm,
	left=40mm,
	right=25mm,
	bottom=30mm,
	headsep=10mm,
	footskip=12mm
} 

\begin{document}

	\begin{titlepage}
		\topmargin12cm
		\ULCornerWallPaper{1}{../Bilder/Heander.jpg}
		\begin{flushright}	
			{\large Masterarbeit \\}
			\begin{Large}
				\textbf{
					Ein System zur partiellen Synchronisation \\ 
					von Wissensbasen für dezentrale soziale Netzwerke \\
				} 
			\end{Large}
			\vspace{1.0cm}
			\begin{large}	
				von Jens Grundmann \\
				\today \\
				\vspace{1.0cm}
				Hochschule für Technik und Wirtschaft Berlin \\
				Fachbereich Wirtschaftswissenschaften II \\
				Studiengang Angewandte Informatik \\
				\vspace{1.0cm}
				Erstgutachter/in Prof. Vorname Name \\
				Zweitgutachter/in Vorname Name \\	
				\vspace{0.5cm}
				\begin{center}
					\includegraphics{../Bilder/htw_logo.jpg}
				\end{center}				
			\end{large}
		\end{flushright}	
	\end{titlepage}
	
	\ClearWallPaper
	\tableofcontents
	\newpage

	\section{Zielsetzung}
	
	Ziel dieser Arbeit ist es eine Softwarekomponente zu entwickeln, die 
	partiellen Synchronisation von Wissensbasen ermöglicht. Eine Wissensbasen
	ist dabei nichts anderes als ein eine Menge an Daten, die in einer
	bestimmten Stuckatur vorliegen bzw. durch eine abstrakte Darstellung
	beschreibbar sind. \\
	Der zu synchronisierende Teilbereich wird als Subspace(engl.:
	\emph{subspace} = Unterraum) bezeichnet in dieser Arbeit, da er einen
	mathematischen Unterraum der Wissensbasis darstellt.
	
	\subsection{Funktionale Anforderungen}
	
	Im Folgendem werden der die funktionalen Anforderungen besprochen. Diese
	beschreiben die Features der Softwarekomponente, welche notwendig sind,
	um das Ziel dieser Arbeit zu erfüllen.
	
	\subparagraph{Beschreibbarkeit}
	
	\begin{itemize}
		\item \textbf{Basis:} Hieraus werden die Daten extrahiert.
		\item \textbf{Kontext:} Beschreibt die SubSpace. Wird zur Extraktion
			  verwendet.
		\item \textbf{Thema:} Grenzt SubSpace von anderer SubSpace ab. Es dient 				  zur (technischen) Identifikation und ist Teil des Kontextes.
	\end{itemize} 	
	
	\subsection{Nicht funktionale Anforderungen}
	
	Die in diesem Teil beschriebenen Eigenschafenten, tragen nicht zur
	Zielerfüllung bei, verbessern aber die Qualität der zu entwickelnden
	Softwarekomponente.
	
	\newpage
	\listoffigures
	\newpage
	\listoftables

\end{document}