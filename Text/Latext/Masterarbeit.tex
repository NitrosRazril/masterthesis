\documentclass[a4paper]{article}  

\usepackage[utf8]{inputenc}
\usepackage[ngerman]{babel}	
\usepackage[babel,german=quotes]{csquotes}
\usepackage[none]{hyphenat}
\usepackage{listings}
\usepackage{graphicx}
\usepackage{float}
\usepackage[hidelinks]{hyperref}
\usepackage{geometry}
\usepackage{wallpaper}
% may add sorting=nty later
\usepackage[backend=bibtex]{biblatex}

\geometry{
	a4paper, 
	top=25mm,
	left=40mm,
	right=25mm,
	bottom=30mm,
	headsep=10mm,
	footskip=12mm
} 

\addbibresource{../Zitate/zitate.bib}

\begin{document}

	\begin{titlepage}
		\topmargin12cm
		\ULCornerWallPaper{1}{../Bilder/Header.jpeg}
		\begin{flushright}	
			{\large Masterarbeit \\}
			\begin{Large}
				\textbf{
					Ein System zur partiellen Synchronisation \\ 
					von Wissensbasen für dezentrale soziale Netzwerke \\
				} 
			\end{Large}
			\vspace{1.0cm}
			\begin{large}	
				von Jens Grundmann \\
				\today \\
				\vspace{1.0cm}
				Hochschule für Technik und Wirtschaft Berlin \\
				Fachbereich Wirtschaftswissenschaften II \\
				Studiengang Angewandte Informatik \\
				\vspace{1.0cm}
				Erstgutachter/in Prof. Vorname Name \\
				Zweitgutachter/in Vorname Name \\	
				\vspace{0.5cm}
				\begin{center}
					\includegraphics{../Bilder/htw_logo.jpg}
				\end{center}				
			\end{large}
		\end{flushright}	
	\end{titlepage}
	
	\ClearWallPaper	
	\newpage
	
	{\LARGE \textbf{Insert later}}
	\begin{itemize}
		\item \hyperref[sec:sharkthemes]{Themen Shark Framework}.
	\end{itemize} 	
	
	\newpage	
	\tableofcontents
	\newpage


	\section{Einleitung}
	
	

	
	Ziel dieser Arbeit ist es eine Softwarekomponente zu entwickeln, die 
	partiellen Synchronisation von Wissensbasen ermöglicht. Eine Wissensbasis
	ist dabei nichts anderes als ein eine Menge an Daten, die in einer
	bestimmten Stuckatur vorliegen bzw. durch eine abstrakte Darstellung
	beschreibbar sind. 
	
	\section{Grundlagen}
	\label{sec:sharkthemes}	
	
	Das folgende Kapitel widmet sich den Grundlagen, auf denen die Arbeit
	aufbaut. Es werden die Themen Shark Framework,[*insert more here*]	
	besprochen.
	
	\subsection{Shark Framework}
	
	Das ist ein Zitat. \cite{SharkFW}
	
	\section{Konzeption}	
	
	\subsection{Funktionale und nicht funktionale Anforderungen}
	
	Im Folgendem werden die funktionalen und nicht funktionale
	Anforderungen besprochen. Diese	beschreiben welche Features die
	Softwarekomponente bereitstellen soll, sowie wichtige Aspekte der
	Qualitätssicherung.
	
	\paragraph{Funktionale Anforderungen}
	\begin{itemize}
		\item \textbf{Beschreibbarkeit:} Es ist möglich einen Raum von
		Daten zu beschreiben und diesen von einem anderen Raum
		von Daten abzugrenzen. 
		\item \textbf{Abhängigkeiten:} Es ist möglich Abhängigkeiten zwischen
		Räumen zu definieren. So soll beispielsweise der Raum Java-Chat ein 
		Kind des Programmiersprachen-Chat Raumes sein können.
		\item \textbf{Persistenz:} Es soll möglich sein die Beschreibung der
		Räume von Daten persistent zu speichern. Die gespeicherten Räume
		bleiben somit erhalten und und können zu späterem Zeitpunkt neu
		geladen werden.
		\item \textbf{Synchronisation:} Es ist möglich die Räume von Daten 
		und ihre Abhängigkeiten mit Peers in einem Peer to Peer Netzwerk 
		zu synchronisieren. Ziel ist es, dass die Räume nach der 
		Synchronisation identisch von Aufbau und Inhalt sind.
	\end{itemize} 	
	
	\paragraph{Nicht funktionale Anforderungen}
	\begin{itemize}
		\item \textbf{Build-Management:} Die Softwarekomponente ist mittels
		eines zu bestimmenden Build Tools so eingerichtet, dass das Aufsetzen 
		der	Entwicklungsumgebung für andere Entwickler schnell und einfach
		zu erledigen ist. Mögliche Synergien des gewählten Tools mit anderen
		Systemen zur Softwareentwicklung und Qualitätssicherung, zum Beispiel  
		Jenkins \cite{Jenkins}, sind wünschenswert.
		\item \textbf{Testbarkeit:} Die zu Softwarekomponente
		ist modular so aufgebaut, dass sie durch Modultest testbar ist.
		\item \textbf{Modultest:} Es existieren bereit eine Reihe von
		Modultest, welche die grundlegende Funktionalität der 
		Softwarekomponente sicherstellen.
	\end{itemize} 
	
	\newpage
	\printbibliography[type=online,heading=subbibliography,title={Webseiten}]
	\newpage
	\listoffigures
	\newpage
	\listoftables

\end{document}